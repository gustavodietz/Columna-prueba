\begin{titlepage} %iniciamos una pagina de titulo
\newcommand{\reglita}{\rule{\linewidth}{0.5mm}}% Este es un ejemplo de lo que se conoce como macro, a un comando de latex le asignamos un nuevo nombre que nos resulte familiar, en este caso se renombra \rule que sirve para dibujar una línea
\center
\includegraphics[width=0.2\textwidth]{pics/logo.png}\\[1cm] % Así añadimos una imagen
%%% Titulo del trabajo
\reglita \\[0.4cm]
{ \huge \bfseries Reducción de Estrés basado en Mindfulness para Dolor de Espalda\\
\Large\bfseries (borrador)}\\[0.4cm]
\reglita \\[2.50cm]
%% nuestros datos
\begin{large}
G.G. and E.A
\end{large}\\[0.4cm]
\vfill %Llenamos con espacios en blanco
\end{titlepage}